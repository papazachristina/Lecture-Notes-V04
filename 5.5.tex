\documentclass[12pt]{book}
\usepackage[margin=.85in]{geometry} % for MARGIN
\usepackage[many]{tcolorbox}    	% for COLORED BOXES (tikz and xcolor included)

\usepackage{multirow}
\usepackage{tabularx}
\usepackage{multicol}   
\usepackage{enumerate}
\usepackage[shortlabels]{enumitem}
\usepackage{varwidth}
\usepackage{tasks}
\usepackage[export]{adjustbox}
\usepackage{array} % For m{} column type

\usepackage{titleps}
\usepackage{setspace}               % for LINE SPACING
\usepackage[⟨options⟩]{fancyhdr}
\usepackage{enumitem}
\setlist{nosep}
\usepackage{tikz}
\usepackage{pgfplots}
\pgfplotsset{compat=1.5.1}
\usetikzlibrary{datavisualization}
\usetikzlibrary{datavisualization.formats.functions}

\newcommand{\D}{\displaystyle}
\newcommand\Mydiv[2]{%
$\strut#1$\kern.25em\smash{\raise.3ex\hbox{$\big)$}}\mkern-8mu
        \overline{\enspace\strut#2}}


\setlength\parindent{0pt}   % killing indentation for all the text
\setstretch{1.3}            % setting line spacing to 1.3
\setlength\columnsep{0.25in} % setting length of column separator
\pagestyle{fancy}           % setting pagestyle to be headings

\usepackage[]{titlesec}

\fancyhead[L]{Math V04 - College Algebra}
\fancyhead[R]{Christina Papazacharioudakis}

\tcbset{
    sharp corners,
    colback = white,
    before skip = 0.2cm,    % add extra space before the box
    after skip = 0.5cm      % add extra space after the box
}                           % setting global options for tcolorbox

    \newtcolorbox{boxR}{
    fontupper = \color{black}, % font color
    boxrule = 1.5pt,
    colframe = black,
    rounded corners,
    arc = 5pt   % corners roundness
}



\begin{document}


{\Large \textbf{5.5 Zeros of Polynomial Functions}}
\vspace{1mm}

In this section, we will discuss a variety of tools for writing polynomial functions and solving polynomial equations. We start this sections off by using polynomial division to help us find the zeros of a polynomial and then writing it's corresponding linear factors.
\vspace{3mm}

{\large \textbf{Using the Factor Theorem to Solve a Polynomial Equation}}


Let's start by dividing $f(x) =x^2-4x-5$ by $x+1$ using synthetic division: 

\vspace{60mm}

Making sense of this division, we have $\D \frac{x^2-4x-5}{x+1}=$

\vspace{10mm}
In light of the Division Algorithm, we can make sense of this division as:
$$f(x) = x^2-4x-5=$$

\vspace{5mm}


In section 5.3, we observed how the zeros of a polynomial related to its linear factors. As a review, let's find the zeros of $f(x)=x^2-4x-5$.
\vspace{45mm}
\begin{boxR}
   \textbf{The Factor Theorem}
    \vspace{1mm}
    \hline
    \vspace{2mm}
    $$ (x-k) \text{ is a factor of } f(x) \hspace{1mm} \text{ if and only if }  k \hspace{1mm} \text{ is a zero of } f(x)$$
\end{boxR}



\newpage



\begin{boxR}
    \textbf{How To}
    \vspace{1mm}
    \hline
    \vspace{2mm}
\textbf{Given a factor and a third-degree polynomial, use the Factor Theorem to factor the polynomial.}

\begin{enumerate}
    \item Use synthetic division to divide the polynomial by $(x-k)$
    \item Confirm that the remainder is $0$.
    \item Write the polynomial as the product of $(x-k)$
  and the quadratic quotient.
    \item If possible, factor the quadratic.
    \item Write the polynomial as the product of factors.  
\end{enumerate}
\end{boxR}

\underline{\textbf{Example 1 - Using the Factor Theorem to Find the Zeros of a Polynomial Expression}}

Show that $(x+2)$ is a factor of $x^3-6x^2-x+30$. Find the remaining factors. Use the factors to determine the zeros of the polynomial.

\newpage


\newpage
{\large \textbf{Using the Rational Zero Theorem to Find Rational Zeros}}

Sometimes the zeros of a polynomial are not always integers. They might be rational numbers. We now look at how to find rational zeros of a polynomial using the \textbf{Rational Zero Theorem}. Let's demonstrate this theorem with an example.

\vspace{3mm}

Consider a quadratic function with two zeros $\D x=\frac{2}{5}$ and $\D x=\frac{3}{4}$. By the Factor Theorem,

The zero $\D x=\frac{2}{5}$ has the factor: 
\vspace{2mm}

The zero $\D x=\frac{3}{4}$ has the factor: 
\vspace{2mm}

Setting each factor to zero: 

\vspace{20mm}

Multiplying each side to eliminate fractions:


\vspace{20mm}

We can now create a quadratic function:

\vspace{20mm}
Expanding the polynomial we have...

\vspace{20mm}
Or...

\vspace{40mm}



We can infer that the \textbf{numerators} of the rational roots will always be factors of the \textbf{constant term} and the \textbf{denominators} will be factors of the \textbf{leading coefficient}. This is the idea of the Rational Zero Theorem; it gives us a pool of possible rational zeros.

\newpage

\begin{boxR}
  \textbf{The Rational Zero Theorem}
    \vspace{1mm}
    \hline
    \vspace{2mm}
Given a polynomial $$\D f(x)=a_nx^n + a_{n-1}x^{n-1} + \ldots + a_1x+a_0$$ with integer coefficients and $a_n \neq 0$, every rational zero of $f(x)$ has the form $\D \frac{p}{q}$ where:
\begin{itemize}
    \item $p$ is a factor of the constant term, $a_0$
    \item $q$ is a factor of the leading term, $a_n$
\end{itemize}
 All possible values of $\frac{p}{q}$ (reduced to lowest terms) create a pool of candidates on what the zeros of a polynomial \textit{could} be.
\end{boxR}


\underline{\textbf{Example 2 - List All Possible Rational Zeros}}

List all possible rational zeros of $f(x)=2x^4-5x^3+x^2-4$.


\newpage

\underline{\textbf{Example 3 - Use the Rational Zero Theorem to Find Rational Zeros}}

Use the Rational Zero Theorem to find the rational zeros of $f(x) = 2x^3 + x^2 - 4x+1$. 



\newpage
{\large \textbf{Finding the Zeros of Polynomial Functions with Synthetic Division}}

Since the \textbf{Rational Zero Theorem} gives us the candidates of possible rational zeros for a polynomial function, we can also use synthetic division to check which ones are the zeros.

\begin{boxR}
  \textbf{  How To}
  \vspace{1mm}
  \hline
  \vspace{2mm}
\textbf{Given a polynomial function, use synthetic division to find its zeros.}
\begin{enumerate}
    \item Use the Rational Zero Theorem to list all possible rational zeros of the function.
    \item Use synthetic division to check if a candidate is a zero.
        \begin{itemize}
             \item  If the remainder is $0$, the candidate is a zero. If the remainder is not zero, discard the candidate.
         \end{itemize}
    \item Repeat step two using the quotient found with synthetic division. If possible, continue until the quotient is a quadratic.
    \item Find the zeros of the quadratic function. Two possible methods for solving quadratics are factoring and using the quadratic formula.
\end{enumerate}
\end{boxR}


\underline{\textbf{Example 4 - Finding the Zeros of a Polynomial Function with Synthetic Division}} 

Find the zeros of $\D f(x)=x^4-2x^3-7x^2+8x+12$.



\newpage
\textbf{ }
\vspace{210mm}

Now let's look at another example following this same process...
\newpage

\underline{\textbf{Example 5 - Finding the Zeros of a Polynomial Function with Synthetic Division}} 

Find the zeros of $f(x)=5x^3+27x^2+11x+5$. 


\newpage
As we just saw, a polynomial $f(x)$ can have complex zeros. This leads to the Fundamental Theorem of Algebra.

\begin{boxR}
    \textbf{The Fundamental Theorem of Algebra}
    \vspace{1mm}
    \hline
    \vspace{2mm}
    If $f(x)$ is a polynomial of degree $n \geq 1$, then $f(x)$ has at least one complex zero.
\end{boxR}

\textbf{Note:} A real number $a$ is also a complex number with an imaginary part of zero: $a=a+0i$. So when we say ``at least one complex number" this could be a real number.

\vspace{3mm}

\begin{boxR}
    \textbf{The Linear Factorization Theorem}
    \vspace{1mm}
    \hline
    \vspace{2mm}
    If $f(x)=a_nx^n + a_{n-1}x^{n-1}+ \ldots +a_1x+a_0$ where $n \geq 1$
and $a_n \neq 0$, then $f(x)$ has exactly $n$ linear factors and can be written as:
    $$f(x)=a_n(x-c_1)(x-c_2)(x-c_3)\cdots (x-c_n)$$

    where $c_1, \ldots c_n$ are complex numbers (possibly real and not necessarily distinct). 
    \end{boxR}





\vspace{10mm}
In Example 4, we found the zeros of $f(x)=5x^3+27x^2+11x+5$ to be:

$5, -5, \hspace{1mm} -\frac{1}{5}+ \frac{2}{5}i, \hspace{1mm} -\frac{1}{5} - \frac{2}{5}i$.
\vspace{5mm}

Referencing the Linear Factorization Theorem, let's re-write $f(x)=5x^3+27x^2+11x+5$ in it's factored form:


\vspace{45mm}

\begin{boxR}
    If $a+bi$ is a zero of a polynomial ($b\neq 0$), then $a-bi$ is also a zero. 
\end{boxR}
\newpage

{\large \textbf{Using the Linear Factorization Theorem to Find Polynomials with Given Zeros}}

Many of our problems so far have involved finding the zeros of a polynomial function that has been given. The Linear Factorization Theorem enables us to reverse this process: find the polynomial function when the zeros are given.  

\vspace{3mm}
\begin{boxR}
    \textbf{How To}
    \vspace{1mm}
    \hline
    \vspace{2mm}
Given the zeros of a polynomial and a point $(c, f(c))$ on its graph, use the Linear Factorization Theorem to find the polynomial function.
\begin{enumerate}
    \item Use the zeros to construct the linear factors of the polynomial.
    \item Multiply the linear factors to expand the polynomial.
    \item Substitute $(c, f(c))$ into the function to determine the leading coefficient, $a_n$.
    \item Simplify.
\end{enumerate}
\end{boxR}

\underline{\textbf{Example 6 - Use the Linear Factorization Theorem to Find a Polynomial}} 

Find the fourth degree polynomial $f(x)$ that has zeros of $-3, -2, i$ such that $f(-2)=100$.



\end{document}


