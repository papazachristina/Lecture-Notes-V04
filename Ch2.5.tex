\documentclass[12pt]{book}
\usepackage[margin=.85in]{geometry} % for MARGIN
\usepackage[many]{tcolorbox}    	% for COLORED BOXES (tikz and xcolor included)


\usepackage{multicol}   
\usepackage{enumerate}
\usepackage[shortlabels]{enumitem}
\usepackage{varwidth}
\usepackage{tasks}
\usepackage[export]{adjustbox}

\usepackage{titleps}
\usepackage{setspace}               % for LINE SPACING
\usepackage[⟨options⟩]{fancyhdr}
\usepackage{enumitem}
\setlist{nosep}
\usepackage{tikz}
\usepackage{pgfplots}
\pgfplotsset{compat=1.5.1}
\usetikzlibrary{datavisualization}
\usetikzlibrary{datavisualization.formats.functions}

\newcommand{\D}{\displaystyle}


\setlength\parindent{0pt}   % killing indentation for all the text
\setstretch{1.3}            % setting line spacing to 1.3
\setlength\columnsep{0.25in} % setting length of column separator
\pagestyle{fancy}           % setting pagestyle to be headings

\usepackage[]{titlesec}

\fancyhead[L]{Math V04 - College Algebra}
\fancyhead[R]{Christina Papazacharioudakis}

\tcbset{
    sharp corners,
    colback = white,
    before skip = 0.2cm,    % add extra space before the box
    after skip = 0.5cm      % add extra space after the box
}                           % setting global options for tcolorbox

    \newtcolorbox{boxR}{
    fontupper = \color{black}, % font color
    boxrule = 1.5pt,
    colframe = black,
    rounded corners,
    arc = 5pt   % corners roundness
}

\definecolor{ballblue}{rgb}{0.13, 0.67, 0.8}

\begin{document}


\textbf{{\Large 2.5 Quadratic Equations}}
\vspace{5mm}

A quadratic equation is an equation containing a second-degree polynomial. For example, $2x^2+3x-1 = 0$ and $x^2-4=0$ are both quadratic equations. The nicest way to solve these is if we can factor, although sometimes, factoring is not possible. We will look at solving quadratic equations in both situations.

\vspace{3mm}
  \begin{boxR}
     \textbf{Quadratic Equations}
    \vspace{1mm}
    \hline
    \vspace{2mm}
A \textbf{quadratic equation} is an equation containing a second-degree polynomial of the form $$ax^2+bx+c=0$$
 
where $a$, $b$, and $c$ are real numbers. If $a \neq 0$, it is in standard form.
\end{boxR}
\vspace{3mm}
{\large \textbf{Solving Quadratic Equations by Factoring}}

In the process of solving quadratic equations by factoring, we use something called the \textbf{zero-product property}.
\vspace{3mm}
    \begin{boxR}
     \textbf{Zero-Product Property}
    \vspace{1mm}
    \hline
    \vspace{2mm}
    $$ \text{If } a \cdot b = 0 \text{ then } a=0 \text{ or } b=0$$
     where $a$ and $b$ are real numbers or algebraic expressions.  
\end{boxR}
\vspace{3mm}

\textbf{Solving Quadratic Expressions with a Leading Coefficient of 1}
\vspace{3mm}

\underline{\textbf{Example 1 - Factor and Solve a Quadratic with a Leading Coefficient of 1}}

Factor and solve the equation: $x^2 + x -6 = 0$.

\newpage

\vspace{3mm}
    \begin{boxR}
     \textbf{How To}
    \vspace{1mm}
    \hline
    \vspace{2mm}
  \textbf{Given a quadratic equation with the leading coefficient of 1, factor it.} 
  \begin{enumerate}
      \item Find two numbers whose product equals c and whose sum equals b.
      \item Use those numbers to write two factors of the form $(x+k)$ or $(x-k)$ where $k$ represents the numbers found in step 1. For example, if the numbers are $1$ and $-2$, the factors are $(x+1)(x-2)$
      \item Solve using the zero-product property by setting each factor equal to zero and solving for the variable.
  \end{enumerate}
\end{boxR}
\vspace{3mm}


\underline{\textbf{Example 3 - Solve a Quadratic Equation Written as a Difference of Squares}}

Solve the difference of squares equation using the zero-product property:  $x^2-9=0$

\vspace{50mm}
\textbf{Solving a Quadratic Equation by Factoring when the Leading Coefficient is not 1}

\vspace{1mm}

\underline{\textbf{Example 4 - Solve a Quadratic Equation Using Grouping}}

Use grouping to factor and solve the quadratic equation: $4x^2+15x+9=0$.


\newpage 
\begin{boxR}
 \textbf{How To}
    \vspace{1mm}
    \hline
    \vspace{2mm}
  \textbf{Given a quadratic equation with the leading coefficient \textbf{not} 1, factor it.} 
  \begin{enumerate}
      \item Multiply $a \cdot c$ from the quadratic equation $ax^2 +bx+c = 0$.
      \item Find two numbers that multiply to $ac$ and add to $b$.
      \item Rewrite the equation by replacing the $bx$ term with two terms using the numbers found in step 2 as the coefficients of $x$.
      \item Factor the first two terms and the last two terms. The expressions in parenthesis should match so that we can factor by grouping. 
      \item Factor out the expression in parentheses. 
      \item Set the expressions equal to zero and solve for the variables. 
  \end{enumerate}
\end{boxR}

\underline{\textbf{Example 5 - Solving a Polynomial of a Higher Degree by Factoring}}

Solve the equation by factoring: $-3x^3-5x^2-2x=0$.
\vspace{50mm}

{\large \textbf{Using the Square Root Property}}

\begin{boxR}
    \textbf{Square Root Property}
     \vspace{1mm}
    \hline
    \vspace{2mm}
    If $x^2 = k$ then $x = \pm \sqrt{k}$ where $k$ is a non-zero real number.
\end{boxR}
\vspace{3mm}

\underline{\textbf{Example 6 - Solving a Quadratic Equation Using the Square Root Property}}

Solve the quadratic equation using the square root property: $x^2=8$
 \newpage

 \underline{\textbf{Example 7 - Solving a Quadratic Equation Using the Square Root Property}}

Solve the quadratic equation using the square root property: $4x^2+1=7$.

\vspace{25mm}
{\large \textbf{Completing the Square}}

Sometimes we run into situations where we cannot factor $ax^2 +bx+c =0$ regardless if $a$ takes on a value of $1$ or not. In these cases, we may use a method called completing the square.

\begin{boxR}
    \textbf{Completing the Square}
    \vspace{1mm}
     \hline
    \vspace{2mm}
    If $x^2+bx$ is a binomial, then adding $\left(\frac{b}{2}\right)^2$, which is the square of half the coefficient of $x$, gives us a trinomial that results into a perfect square: 

    $$ x^2 +bx + \left(\frac{b}{2}\right)^2 = \left(x+\frac{b}{2}\right)^2$$
\end{boxR}
A visual: 

\vspace{45mm}
 \underline{\textbf{Example 8 - Solving a Quadratic By Completing the Square}}.

Solve the quadratic equation by completing the square: $x^2-3x-5=0$. 

\newpage

\begin{boxR}
    \textbf{How To}
    \vspace{1mm}
     \hline
    \vspace{2mm}
    \textbf{Solve a Quadratic by Completing the Square}

    \begin{enumerate}
        \item Move the constant term to the right side of the equal sign.
        \item Take $\frac{1}{2}$ of the $b$ term and square it. 
        \item Add the result to both sides of the equal sign. 
        \item Rewrite the trinomial as a perfect square. 
        \item Simplify the right side. 
        \item Use the square root property and solve.
    \end{enumerate}
\end{boxR}
\vspace{3mm}

{\large \textbf{Using the Quadratic Formula}}
\vspace{3mm}

The fourth method of solving a quadratic equation is by using the quadratic formula to solve quadratic equations. 

\vspace{3mm}
\begin{boxR}
    \textbf{The Quadratic Formula}
    \vspace{1mm}
    \hline
    \vspace{2mm}
    Any quadratic equation, $ax^2+bx+c=0$, can be solved using the \textbf{quadratic formula}:
    $$ x = \frac{-b \pm \sqrt{b^2-4ac}}{2a}$$
    where $a$, $b$, and $c$ are real numbers and $a\neq 0$.
    
\end{boxR}

Although the quadratic formula works on any quadratic equation in standard form, it is easy to make errors in substituting the values into the formula. Pay close attention when substituting, and use parentheses when inserting a negative number.
\vspace{3mm}

 \underline{\textbf{Example 9 - Solve the Quadratic Equation Using the Quadratic Formula}}.

Solve the quadratic equation $x^2+5x+1=0$.


\newpage

\underline{\textbf{Example 9 - Solve the Quadratic Equation Using the Quadratic Formula}}.

Solve the quadratic equation $x^2+5x+1=0$.

\vspace{60mm}

\begin{comment}
    {\large \textbf{The Discriminant}}
The quadratic formula not only gives us solutions, but it also tells us the nature of the solutions when we focus on what's under the square root: $b^2-4ac$. 

\begin{boxR}
    \textbf{The Discriminant}
    \vspace{1mm}
    \hline
    \vspace{2mm}
    The \textbf{discriminant} of $ax^2 +bx+c = 0$ is $$b^2-4ac$$ which is the expression under the square root in the quadratic formula:
    
    $\D x = \frac{-b \pm \sqrt{b^2-4ac}}{2a}$
\end{boxR}
\end{comment}



\end{document}





