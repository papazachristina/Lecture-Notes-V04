\documentclass[12pt]{book}
\usepackage[margin=.85in]{geometry} % for MARGIN
\usepackage[many]{tcolorbox}    	% for COLORED BOXES (tikz and xcolor included)


\usepackage{multicol}   
\usepackage{enumerate}
\usepackage[shortlabels]{enumitem}
\usepackage{varwidth}
\usepackage{tasks}
\usepackage[export]{adjustbox}

\usepackage{titleps}
\usepackage{setspace}               % for LINE SPACING
\usepackage[⟨options⟩]{fancyhdr}
\usepackage{enumitem}
\setlist{nosep}
\usepackage{tikz}
\usepackage{pgfplots}
\pgfplotsset{compat=1.5.1}
\usetikzlibrary{datavisualization}
\usetikzlibrary{datavisualization.formats.functions}

\newcommand{\D}{\displaystyle}


\setlength\parindent{0pt}   % killing indentation for all the text
\setstretch{1.3}            % setting line spacing to 1.3
\setlength\columnsep{0.25in} % setting length of column separator
\pagestyle{fancy}           % setting pagestyle to be headings

\usepackage[]{titlesec}

\fancyhead[L]{Math V04 - College Algebra}
\fancyhead[R]{Christina Papazacharioudakis}

\tcbset{
    sharp corners,
    colback = white,
    before skip = 0.2cm,    % add extra space before the box
    after skip = 0.5cm      % add extra space after the box
}                           % setting global options for tcolorbox

    \newtcolorbox{boxR}{
    fontupper = \color{black}, % font color
    boxrule = 1.5pt,
    colframe = black,
    rounded corners,
    arc = 5pt   % corners roundness
}

\definecolor{ballblue}{rgb}{0.13, 0.67, 0.8}

\begin{document}


{\Large \textbf{2.6 Other Types of Equations}}

We have solved linear equations, rational equation that resulted in linear equations, and quadratic equations using several methods. However, there are many other types of equations, and we will investigate a few more types in this section.

\vspace{3mm}

{\large \textbf{Solving Equations Involving Rational Exponents}}
\vspace{3mm}
\begin{boxR}
    \textbf{Rational Exponents}
    \vspace{1mm}
    \hline
    \vspace{2mm}
   A rational exponent indicates a power in the numerator and a root in the denominator. There are multiple ways of writing an expression, a variable, or a number with a rational exponent using exponent rules:
   \begin{align*}
       a^{\frac{m}{n}} &= \left(a^\frac{1}{n}\right)^m =  \left(\sqrt[n]{a}\right)^m \\
       a^{\frac{m}{n}} &= \left(a^m\right)^\frac{1}{n} =\sqrt[n]{a^m}
   \end{align*}
\end{boxR}

\underline{\textbf{Example 1 - Evaluating a Number Raised to a Rational Exponent}}.
\vspace{1mm}

Evaluate $8^\frac{2}{3}$.

\vspace{25mm}

\underline{\textbf{Example 2 - Solve the Equation Involving Rational Exponents}}.
\vspace{1mm}


Solve the equation $\D x^{\frac{5}{4}}=32$.
\vspace{25mm}


\underline{\textbf{Example 3 - Solve the Equation Involving Rational Exponents and Factoring}}
\vspace{1mm}

Solve $3x^{\frac{3}{4}} = x^\frac{1}{2}$.

\newpage
\textbf{\large Solving Equations Using Factoring}
\vspace{3mm}

We have used factoring to solve quadratic equations, but it is a technique that we can use with many types of polynomial equations. Lets take a look!
\\

\underline{\textbf{Example 4 - Solving a Polynomial by Factoring}}

Solve the polynomial by factoring: $5x^4 = 80x^2$.
\vspace{35mm}


\underline{\textbf{Example 5 - Solving a Polynomial by Grouping}}

Solve the polynomial by grouping: $x^3+x^2-9x-9=0$.
\vspace{35mm}


{\large \textbf{Solving Radical Equations}}

\textbf{Radical Equations} are equations that contain variables in the radicand (the expression under the radical symbol). Here are some examples: 
\begin{align*}
    \sqrt{3x+8} &= x \\
    \sqrt{x+3} &= x-3 \\
    \sqrt{x+5} - \sqrt{x-3} &= 2 
\end{align*}

\newpage

\begin{boxR}
    \textbf{How To}
    \vspace{1mm}
    \hline
    \vspace{2mm}
    \textbf{Given a radical equation, solve it.}
    \begin{enumerate}
        \item Isolate the radical expression on one side of the equal sign. Put all remaining terms on the other side. 
        \item If the radical is an nth root, raise both sides to the nth power. For example, if it is a square root, then square both sides. If it is a cube root, cube both sides.
        \item Solve the remaining equation. 
        \item If a radical term still remains, repeat steps 1-2.
        \item Confirm solutions by substituting into the original equation.
    \end{enumerate}
\end{boxR}

\underline{\textbf{Example 6 - Solving an Equation with One Radical}}
\vspace{1mm}

Solve $\sqrt{15-2x}=x$.

\vspace{40mm}


\underline{\textbf{Example 7 - Solving an Equation with Two Radicals}}
\vspace{1mm}

Solve $\sqrt{2x+3} + \sqrt{x-2}= 4$.

\newpage
{\large \textbf{Solving an Absolute Value Equation}}

Next, we will learn how to solve an absolute value equation. Before we look at absolute value equations, let's look at the absolute value function and what it does. 
\vspace{3mm}
\begin{boxR}
The Absolute Value
    \vspace{1mm}
    \hline
    \vspace{2mm}
    The \textbf{absolute value of} $\mathbf{x}$ is written as $\mathbf{|x|}$. It has the following properties: 
\begin{align*}
    \text{If } x &\geq 0, \text{ then } |x| = x. \\
    \text{If } x &< 0, \text{ then } |x| = -x.
\end{align*}
For real numbers A and B, an equation of the form $|A|=B$ with $B \geq 0$ will have solutions when $A=B$ or $A=-B$. If $B < 0$, then the equation $|A|=B$ has no solution. 
  

\end{boxR}
Let's see what this means:

\vspace{40mm}
\begin{boxR}
    

An \textbf{absolute value equation} in the form $|ax+b|=c$ has the following properties:
  has the following properties:
\begin{align*}
    \text{If } c<0, \hspace{3mm} |ax+b| = c \text{ has no solution. }  \\
    \text{If } c=0, \hspace{3mm} |ax+b| = c \text{ has one solution. }  \\
    \text{If } c>0, \hspace{3mm} |ax+b| = c \text{ has two solutions. } 
\end{align*}
\end{boxR}

\newpage
\underline{\textbf{Example 8 - Solve Absolute Value Equations}}
\vspace{1mm}

Solve the absolute value equations.

  \begin{enumerate}[(a)]
    \item $|6x+4|=8$
    \vspace{60mm}
    \item $|3x+4|=-9$ 
    \vspace{30mm}
    \item $|3x-5|-4=6$
    \vspace{60mm}
    \item $|-5x+10|=0$
    \vspace{20mm}
    \end{enumerate}


\newpage
{\large \textbf{Solving Equations in Quadratic Form}}
\vspace{3mm}

As a reminder, a quadratic equation is an equation of the form $ax^2 +bx+c=0$. 
\vspace{3mm}

\textbf{Equations in quadratic form} are equations that follow a similar pattern: the the leading term has a power that is not not always two, but it is double the exponent of the middle term. Lets look at some examples of equations in quadratic form: 

\vspace{40mm}

\begin{boxR}
\textbf{How To}
 \vspace{1mm}
    \hline
    \vspace{2mm}
    \textbf{Given an equation in quadratic form, solve it.}
    \begin{enumerate}
        \item Identify the exponent on the leading term and determine whether it is double the exponent on the middle term.
        \item Assuming it is, substitute a variable, such as $u$, for the variable portion of the middle term. 
        \item Rewrite the equation (with $u$ or other variable) so that it takes on the standard form of a quadratic.
        \item Solve using one of the usual methods for solving a quadratic.
        \item Replace the substitution variable with the original term.
        \item Solve the remaining equation.
    \end{enumerate}
\end{boxR}
\newpage
\underline{\textbf{Example 9 - Solving a Higher Degree Equation in Quadratic Form}}

Solve this fourth degree equation: $x^4-8x^2-9=0$

\vspace{75mm}
\underline{\textbf{Example 10 - Solving an Equation in Quadratic Form Containing a Binomial}}

Solve this fourth degree equation: $(x+2)^2+11(x+2)-12=0$

\newpage
{\large \textbf{Solving Equations in Quadratic Form}}
\vspace{3mm}

Earlier, we solved rational equations that resulted in solving a linear equation one we cleared the denominator with an LCD (section 2.2). Now we look at solving rational equations that result to a quadratic.

\begin{boxR}
    \textbf{Given a rational equation, solve it. }
    \vspace{1mm}
    \hline
    \vspace{2mm}
    \begin{enumerate}
        \item Factor all denominators in the equation.
        \item Find the LCD
        \item Multiply the whole equation through by the LCD. If the LCD is correct, there will be no denominators left. 
        \item Solve the remaining equation (linear or quadratic).
        \item Make sure to check solutions back in the original equation.
    \end{enumerate}
\end{boxR}

\underline{\textbf{Example 11 - Solving Rational Equation Leading to a Quadratic}}
\vspace{3mm}

Solve the following rational equation: $\D \frac{-4x}{x-1} + \frac{4}{x+1} = \frac{-8}{x^2-1}$.
\end{document}